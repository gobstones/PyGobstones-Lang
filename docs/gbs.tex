%
% Copyright (C) 2011, 2012 Pablo Barenbaum <foones@gmail.com>
%
% This program is free software: you can redistribute it and/or modify
% it under the terms of the GNU General Public License as published by
% the Free Software Foundation, either version 3 of the License, or
% (at your option) any later version.
%
% This program is distributed in the hope that it will be useful,
% but WITHOUT ANY WARRANTY; without even the implied warranty of
% MERCHANTABILITY or FITNESS FOR A PARTICULAR PURPOSE. See the
% GNU General Public License for more details.
%
% You should have received a copy of the GNU General Public License
% along with this program. If not, see <http://www.gnu.org/licenses/>.
%

\documentclass{article}
\usepackage[spanish]{babel}
\usepackage[utf8]{inputenc}
\usepackage{xspace,underscore,color}

\hyphenation{parser}
\hyphenation{mo-di-fi-can-do}
\hyphenation{tokens}


\hyphenation{Gobstones}
\hyphenation{PyGobstones}

\newcommand{\PGbs}{PyGobstones\xspace}
\newcommand{\Version}{0.9.2}
\newcommand{\Gbs}{Gobstones\xspace}
\newcommand{\GVM}{\textsc{PyGvm}\xspace}
\newcommand{\gbs}[1]{\texttt{#1}}
\newcommand{\py}[1]{\texttt{#1}}
\newcommand{\arch}[1]{\texttt{#1}}
\newcommand{\en}[1]{{\em{#1}\xspace}}
\newcommand{\Ejemplo}{\subsubsection*{Ejemplo}}
\newcommand{\subsubsubsection}[1]{\vspace{10pt}\noindent\textit{#1}\\}
\newcommand{\enf}[1]{{\textbf{#1}\xspace}}

\newcommand{\TODO}[1]{\textcolor{red}{#1}}



\newcommand{\fromImportClause}{\texttt{from/import}\xspace}

\begin{document}

\begin{center}
\Large{\PGbs}\\
\vspace{10pt}\large{{\em Hacker's guide}}
\end{center}

\tableofcontents

\section{Introducci\'on}

\PGbs es una implementaci\'on del lenguaje \Gbs desarrollada en Python.
Este documento pretende ser una gu\'ia para entender y modificar
el c\'odigo fuente en Python.
El int\'erprete y entorno de desarrollo \PGbs consta de los siguientes m\'odulos:

\begin{itemize}
\item \py{common}: funciones y clases usadas en todo el proyecto.
\item \py{lang}: compilador, analizador y \en{runtime} del lenguaje.
\item \py{gui}: interfaz gr\'afica.
\end{itemize}

\section{M\'odulo \py{common}}

Este m\'odulo implementa aspectos transversales al
proyecto:

\begin{itemize}
\item \py{common.utils} contiene funciones y clases de uso general.

\item \py{common.position} define clases necesarias para
hacer \en{tracking} de las posiciones en los archivos
fuente (esencial para el reporte de errores). Define las clases:

\begin{itemize}
\item \py{Position} --
representa una posici\'on dentro de un programa
(dado por un \en{string}).
Mantiene el n\'umero de fila y columna en el archivo.
\item \py{ProgramElement} --
representa un fragmento de un programa (determinado por las posiciones
donde empieza y termina el fragmento).
Es una clase abstracta. Los tokens y los nodos del AST son las
subclases que la implementan.
\item \py{ProgramArea}, \py{ProgramAreaNear} --
representan un \'area del programa.
Concretamente, \py{ProgramAreaNear(elem)}
es el \'area que circunda un \py{ProgramElement},
usado para que el reporte de errores se vea ``lindo''.
\end{itemize}

\item \py{common.i18n} -- m\'odulo de internacionalizaci\'on.
Es b\'asicamente un diccionario de \en{strings} en \en{strings}.
Todas las cadenas de texto en lenguaje natural que ocurren
en alguna parte del programa est\'an \en{wrappeadas} por un
llamado a \py{common.i18n.i18n}. Por ejemplo,
en el c\'odigo se usa \py{i18n('help')}
y en tiempo de ejecuci\'on se traduce a \py{'ayuda'}.
Por ahora los cambios de idioma se hacen a mano,
modificando en este archivo cu\'al es el ``locale'' (diccionario)
vigente.

Siempre que se quiera agregar un nuevo mensaje (error,
ayuda, opci\'on de men\'u, etc.) se debe agregar una entrada en
este archivo. Los nombres de las palabras reservadas,
operaciones \en{built-in} y otros conceptos del lenguaje
tambi\'en est\'an siempre considerados como meritorios de
traducci\'on.
\end{itemize}

En \py{common.utils} se define la
clase \py{SourceException}.
Casi todas las excepciones definidas en el proyecto heredan
de esta clase. Una \py{SourceException} tiene un mensaje de error,
pero adem\'as tiene asociada un \'area dentro del programa
(\py{common.position.ProgramArea}). Por ejemplo, si el
\en{parser} identifica un error de sintaxis, podr\'ia lanzar una
excepci\'on como:

\py{SourceException('error de sintaxis', ProgramAreaNear(token))} 

%\Ejemplo
%Las posiciones se usan escaneando un \en{string}.
%Por ejemplo \verb|Position('a b\nc d')| crea la posici\'on
%al comienzo de la cadena dada. El argumento \py{filename}
%es s\'olo indicativo, no est\'a realmente asociado a un
%archivo (si se omite se usa ``...'').
%\begin{verbatim}
%>>> import common.position
%>>> p = common.position.Position('a b\nc d', filename='entrada.txt')
%>>> p
%entrada.txt:1:1
%>>> p.after_reading('a b')
%entrada.txt:1:4
%>>> p.after_reading('a b').after_reading('\nc')
%entrada.txt:2:2
%\end{verbatim}

\section{M\'odulo \py{lang}}

En este m\'odulo se implementa el lenguaje \Gbs.
El procedimiento general para ejecutar un programa \Gbs es
el t\'ipico:

\begin{itemize}
\item An\'alisis sint\'actico -- a partir de la
cual se genera un \'arbol de sintaxis abstracta (AST).
\item An\'alisis est\'atico -- incluye varias funciones
que procesan recursivamente el AST:
\begin{itemize}
\item \en{lint} -- a medio camino
entre el an\'alisis sint\'actico y el sem\'antico,
chequea algunas restricciones \enf{no} opcionales
del lenguaje. Esta fase es precondici\'on para la compilaci\'on
correcta.
\item An\'alisis de variables vivas -- identifica usos
de variables potencialmente no declaradas, y definiciones
de variables nunca usadas. Esta fase es opcional.
\item Inferencia de tipos -- esta fase tambi\'en es
opcional.\footnote{En la GUI, la opci\'on ``Ejecutar'' no
aplica las fases opcionales de an\'alisis, pero la
opci\'on ``Chequear'' s\'i.}
\end{itemize}

\item Compilaci\'on -- dado un AST, genera c\'odigo para la
m\'aquina virtual {\em ad hoc} de \PGbs (\GVM).

\item Interpretaci\'on de la m\'aquina virtual -- interpreta
c\'odigo para la \GVM.
\end{itemize}

La manera de indicar un error, en cualquier punto del \en{pipeline},
es elevando una excepci\'on, instancia de \py{common.utils.SourceException}
o sus subclases. Esto no s\'olo describe el error con un mensaje,
sino que indica el \'area precisa del programa en la que se
identific\'o el error.

\subsection{An\'alisis sint\'actico}

El an\'alisis sint\'actico se divide conceptualmente en las fases
usuales de tokeni\-zaci\'on y \en{parsing}. Los m\'odulos asociados
al analizador sint\'actico son:
\begin{itemize}
\item \py{lang.bnf_parser} -- parser gen\'erico.
\item \py{lang.ast} -- funciones asociadas al \'arbol de sintaxis.
\item \py{lang.gbs_parser} -- parser de \Gbs.
\end{itemize}

\subsubsection{\py{lang.bnf_parser}}

Este m\'odulo implementa un tokenizador y
generador de parsers LL(1) a partir de una gram\'atica
arbitraria. Define las clases:
\begin{itemize}
\item \py{Token} -- subclase de \py{common.position.ProgramElement},
      representa un fragmento del programa donde ocurre un
      cierto s\'imbolo terminal de la gram\'atica.
      El s\'imbolo terminal (\py{tok.type}) es
      conceptualmente un tipo
      enumerado, y se representa con una cadena.
      El valor puede ser un objeto arbitrario de Python.
      Por ejemplo, \verb|Token('num', 42, pos1, pos2)|
      podr\'ia representar una ocurrencia del terminal ``42''
      interpretado como numeral.
\item \py{Lexer} -- representa un analizador l\'exico.
      Se inicializa con una tabla de tokenizaci\'on de la
      forma:\\
       \verb|[(tipo1, regexp1), ..., (tipoN, regexpN)]|\\
      donde $tipo_i$ es una cadena (el s\'imbolo terminal) y
      $regexp_i$ es una expresi\'on regular.

      El m\'etodo \py{tokenize} toma una cadena y genera
      el \en{stream} de \py{Token}s corres\-pon\-dien\-te.
      La tokenizaci\'on es golosa, respetando el orden
      de aparici\'on en la tabla.

      Se hacen algunas consideraciones ``m\'agicas'' para ignorar ciertos
      \en{tokens} (blancos y comentarios) e identificar con un s\'imbolo
      terminal \'unico a las palabras reservadas.

\item \py{Parser} -- representa un analizador sint\'actico.
      Se inicializa con una gram\'atica. La gram\'atica es
      un diccionario que, dado un no terminal, lo asocia a una
      lista de producciones.
      Una producci\'on es esencialmente una lista de s\'imbolos
      (no terminales y terminales), potencialmente asociada a
      una acci\'on.
      A partir de esto se arma la tabla
      de \en{parsing} LL(1).\footnote{
      Algoritmo de an\'alisis sint\'actico descendente de la Secci\'on 4.4 de [1].}

      El m\'etodo \py{parse} toma un \en{stream} de \py{Token}s
      y genera un \en{stream} de eventos de \en{parsing}.
      Un evento de \en{parsing} puede ser de una de las dos formas siguientes:
      \begin{itemize}
      \item \verb|'PRODUCE', no_terminal, producción|
      \item \verb|'CONSUME', terminal,    token|
      \end{itemize}
      El primero representa la expansi\'on de un no terminal
      en una cierta producci\'on, y el segundo el consumo de
      un token de la entrada. A partir de estos eventos se
      puede generar el \'arbol de an\'alisis sint\'actico.

\item \py{Analyzer} -- combina las clases \py{Lexer} y \py{Parser}.
      Se inicializa con un archivo que contiene una gram\'atica
      en formato BNF. A partir de esta gram\'atica, construye
      la tabla de tokenizaci\'on, y el diccionario de producciones,
      con las cuales inicializa el \en{lexer} y el \en{parser}.
      Ver el Ap\'endice A para un comentario sobre el formato
      de la gram\'atica.

      El m\'etodo \py{parse} toma una cadena y genera un
      \en{stream} de eventos de \en{parsing}.
\end{itemize}

\subsubsection{\py{lang.ast}}

Este m\'odulo implementa las clases:
\begin{itemize}
\item \py{ASTNode} --
subclase de \py{common.position.ProgramElement},
que representa un nodo del \'arbol de sintaxis abstracta.
Un nodo del AST consta de una lista de objetos,
algunos de los cuales pueden ser subASTs.
Adem\'as, por ser un \py{ProgramElement}, un nodo
del AST conoce las posiciones del c\'odigo fuente en las
que empieza y termina el fragmento del programa
que le dio lugar.

\item \py{ASTBuilder} -- clase encargada de construir un AST.

El m\'etodo \py{build_ast_from} toma un \en{stream} de
eventos de \en{parsing} (ge\-ne\-ra\-dos por un \py{lang.bnf_parser.Parser})
y construye el AST asociado.\footnote{
En este caso no basta con una simple funci\'on recursiva
porque la implementaci\'on natural requiere mantener estado
global (una pila).}

Notar que el \'arbol asociado a los eventos es concreto y
no abstracto. Para construir el \'arbol abstracto, en
\py{lang.ast} se implementa tambi\'en la eva\-lua\-ci\'on de las acciones
de la gram\'atica LL(1). Esto permite implementar acciones del estilo
habitual en generadores de parsers como \texttt{yacc}:
\begin{verbatim}
    if ( <gexp> ) <blockcmd> <else?>  @ if $3 $5 $6
\end{verbatim}
En este caso la producci\'on es lo que est\'a antes de la arroba (\verb|@|)
y la acci\'on asociada lo que est\'a despu\'es.
Cuando se terminan de consumir los s\'imbolos terminales de la
producci\'on (las hojas de toda la derivaci\'on),
se construye el AST etiquetado con el no terminal \verb|if| en
la ra\'iz, y los tres subASTs sintetizados por \verb|<gexp>|,
\verb|<blockcmd>| y \verb|<else?>|.

Se implementan tambi\'en algunas pseudoacciones que,
adem\'as de construir el nodo del AST, lo manipulan
de alguna manera arbitraria. En particular,
se utiliza la pseudoacci\'on \py{INFIXL} para introducir operadores asociativos
a izquierda, lo cual es complicado en gram\'aticas LL(1),
ya que las reglas recursivas lo son a derecha.
\end{itemize}

\subsubsection{Gram\'atica de \Gbs}

La gram\'atica de \Gbs en BNF est\'a ubicada en \arch{lang/gbs_grammar.bnf}
Est\'a adaptada de la especificaci\'on oficial, con ligeros
cambios para que las reglas sean recursivas a derecha, y para
evitar algunas ambig\"uedades propias del \en{parsing} LL(1).
Adem\'as, est\'a extendida con acciones que permiten generar
el AST.

Algunas restricciones sint\'acticas que oficialmente son parte de
la BNF se postergan hasta la fase de \en{lint}. Por ejemplo, la
gram\'atica BNF dada en \arch{gbs_grammar.txt} fusiona el
cuerpo de las funciones y el cuerpo de los procedimientos
en un \'unico s\'imbolo no terminal, y permite que ambos
tengan (o no tengan) un \gbs{return}.

En la especificaci\'on oficial, los procedimientos, funciones
y constantes reservados son parte de la sintaxis. En la
implementaci\'on de \PGbs, son procedimientos, funciones
y constantes comunes y corrientes (\en{built-in}).

Este tipo de verificaciones podr\'ia hacerse directamente
desde la BNF, como en la especificaci\'on oficial, pero se lo hace
en un paso separado, lo cual simplifica la gram\'atica y
permite reportar los errores de manera m\'as clara mediante
una funci\'on especializada (la fase de \en{lint}).

\subsubsection{\py{lang.gbs_parser}}

En este m\'odulo se definen las clases \py{GbsLexer} y
\py{GbsParser}, subclases de \py{lang.bnf_parser.Lexer}
y \py{lang.bnf_parser.Parser} respectivamente.
Estas clases hacen lo mismo que sus superclases,
agregando c\'odigo para identificar errores comunes
y particulares de la gram\'atica de \Gbs seg\'un
figura en la BNF descripta en la secci\'on anterior.

M\'as all\'a del chequeo de errores, que no es m\'as que
un mont\'on de \py{if}s, este m\'odulo resuelve una
ambig\"uedad presente en la gram\'atica LL(1) de \Gbs.

El problema es que \Gbs (aparentemente) \enf{no} es LL(1) porque hay
cadenas, como la siguiente, que son conflictivas con un
\en{token} de \en{lookahead}:
\begin{verbatim}
    a := f
    (x) := g(y)
\end{verbatim}
Cuando se encuentra con \verb|f|, el \en{parser} tiene que
determinar si \gbs{f} es una funci\'on y el par\'entesis que
le sigue corresponde a su lista de argumentos, o bien si
\gbs{f} es una variable y el par\'entesis que le sigue
corresponde a una asignaci\'on de tuplas.

En una situaci\'on como esta, es claro que la \'unica opci\'on v\'alida
es la segunda, pero el \en{parser} no puede determinarlo con
un solo \en{token} de \en{lookahead} y sin hacer \en{backtracking}.
Esto da lugar a un conflicto \en{first/follow}.

La gram\'atica BNF est\'a escrita de tal manera que
el conflicto se resuelva desambiguando siempre por
el primer caso; es decir: si \gbs{f} est\'a seguida
por un par\'entesis, se considera que \gbs{f} es una
funci\'on acompa\~nada por su lista de argumentos.

Para resolver el conflicto, se recurre a un
truco a nivel l\'exico. Cuando el \en{lexer} identifica una situaci\'on
como la de arriba, hace el \en{lookahead}
que sea necesario para determinar si \gbs{f} deber\'ia ser
una variable o no. En tal caso, el \en{lexer} produce como
\en{token} adicional un punto y coma que desambigua la
situaci\'on:

\begin{verbatim}
    a := f ;
    (x) := g(y)
\end{verbatim}
Lo que hace el \en{lexer} en este caso, para resolver la ambig\"uedad,
es leer toda la tupla que le sigue a \gbs{f} (i.e. una lista de \en{tokens}
con par\'entesis balanceados, hasta que se cierra el par\'entesis inicial).
Si el \en{token} que le sigue a la tupla es
un \verb|:=|, se trata de una asignaci\'on de tuplas y hay
que insertar un punto y coma. De lo contrario, se trata de
un llamado a funci\'on y no hay nada que hacer.

La resoluci\'on de esta ambig\"uedad est\'a
implementada en el m\'etodo:\\
\py{lang.gbs_parser.GbsLexer._tokenize_solve_conflict}


\subsection{An\'alisis est\'atico}

Una vez construido el AST, se obtiene un objeto de la forma
especificada en el Ap\'endice B.

\subsubsection{Representaci\'on de tipos y valores}

Para el an\'alisis sem\'antico del programa, es
necesario introducir representaciones para
algunas entidades del lenguaje.

\begin{itemize}
\item \py{lang.gbs_constructs} -- define clases que
representan varias cons\-truc\-cio\-nes de \Gbs:
procedimientos, funciones, constantes y varia\-bles,
ya sean definidos por el usuario o \en{built-in}.
Se usan para representar qu\'e tipo de valor
est\'a asociado a un cierto identificador.

Por ejemplo,
el identificador \gbs{Rojo} estar\'ia asociado a la clase
\py{BuiltinConstant}, mientras
un procedimiento como \gbs{IrAlExtremo} estar\'ia asociado
a la clase \py{UserProcedure}.

\item \py{lang.gbs_type} -- define clases que representan
los tipos de datos en \Gbs. Los tipos de datos admitidos
son los siguientes:
  \begin{itemize}
  \item Tipos b\'asicos:\\ \gbs{Bool}, \gbs{Int}, \gbs{Dir}, \gbs{Color}
  \item Procedimientos y funciones:\\
        \verb|procedure (t_1, ..., t_n)|\\
        \verb|function (t_1, ..., t_n) -> (s_1, ..., s_m)|\\
        con $n, m > 0$ donde los \gbs{t_i} y los \gbs{s_i} son tipos
        b\'asicos.
  \item Variables de tipos: se representan con una referencia
        (indirecci\'on), como es usual en las implementaciones de
        Prolog.
        Si la indirecci\'on es \py{None}, la variable no est\'a
        instanciada. En otro caso, la variable est\'a instanciada
        en el tipo referenciado (que a su vez puede ser una indirecci\'on).
  \item Tipos polim\'orficos: se introducen por funciones como
        \gbs{siguiente}, que pueden trabajar con valores de cualquiera
        de los cuatro tipos b\'asicos.
  \end{itemize}
Adem\'as de las clases para representar tipos,
el m\'odulo provee la funci\'on \py{unify}, que toma
dos tipos y los unifica. Si la unificaci\'on falla,
se eleva una excepci\'on. La unificaci\'on act\'ua
instanciando (mutando) las variables a\'un no instanciadas.\footnote{
Algoritmo de unificaci\'on de la Secci\'on 6.7 de [1].}

El m\'odulo tambi\'en provee la funci\'on \py{parse_type_declaration};
esta funci\'on toma un AST que describe un tipo y construye un objeto
que lo representa.

\item \py{lang.gbs_builtins} -- en este m\'odulo se definen todos los
valores provistos nativamente por \Gbs. Esto incluye los valores asociados a
los cua\-tro colores y las cuatro direcciones, los enteros y booleanos.
Tambi\'en incluye las funciones \en{built-in} que manipulan todos estos
tipos (\gbs{previo}, \gbs{siguiente}, \gbs{opuesto}, etc.), los procedimientos
que manejan el tablero (\gbs{Poner}, \gbs{Sacar}, \gbs{Mover}, etc.).

Cada valor \en{built-in} se define mediante: el nombre global al que
estar\'a li\-ga\-do, su tipo y la representaci\'on interna de su valor.
Las funciones y procedimientos \en{built-in} se representan con
funciones de Python que toman un estado global y un conjunto de
par\'ametros, pudiendo eventualmente modificar el estado global.
La representaci\'on del tablero de \Gbs se mantiene separada, en
el m\'odulo \py{lang.gbs_board}.
\end{itemize}

\subsubsection{Fase de \en{lint}}

El m\'odulo \py{lang.gbs_lint} define la funci\'on
\py{lint}, encargada de veri\-fi\-car en una primera
instancia la correcci\'on sem\'antica y sint\'actica de un programa \Gbs.
La gram\'atica BNF permite ciertas libertades en el programa fuente
que no est\'an admitidas en la sintaxis oficial de \Gbs.
La funci\'on toma el AST de un programa y revisa:
  \begin{itemize}
  \item[-] que el \gbs{procedure Main} sea la \'ultima definici\'on de un programa;
  \item[-] que no haya procedimientos ni funciones definidos dos veces
           (esto incluye declaraciones de rutinas en cl\'ausulas \fromImportClause);
  \item[-] que las funciones empiecen con min\'uscula y los procedimientos
           con may\'uscula;
  \item[-] que los \'indices de un \gbs{repeatWith} y par\'ametros
           de un procedimiento o funci\'on no se mezclen entre s\'i,
           ni se asignen; tambi\'en se proh\'ibe que haya \'indices
           de un \gbs{repeatWith} anidados con el mismo nombre;
  \item[-] que el cuerpo de una funci\'on tenga un \gbs{return};
  \item[-] que el cuerpo de un procedimiento no tenga un \gbs{return},
           a menos que se trate del \gbs{Main} (y en tal caso, que
           devuelva s\'olo variables);
  \item[-] que no haya par\'ametros repetidos en la lista de
           par\'ametros formales de una funci\'on o procedimiento;
  \item[-] que los nombres de variables y funciones no se mezclen;\footnote{
           Esta decisi\'on es discutible, no es obligatoria en la versi\'on
           oficial de \Gbs, e incluso es delicada por cuestiones de $\alpha$-conversi\'on.
           Pero dado que el lenguaje est\'a pensado con fines did\'acticos
           y para programas cortos, esta restricci\'on adicional
           puede ayudar.}
  \item[-] que no haya variables repetidas en el lado izquierdo
           de una asignaci\'on a una tupla;
  \item[-] que los literales en las ramas de un case sean disjuntos;
  \item[-] que todos los usos de procedimientos y funciones
           coincidan con la aridad declarada, tanto en la
           cantidad de argumentos recibidos, como en la cantidad
           de valores devueltos por una funci\'on al asignar
           a una tupla;
  \item[-] que los m\'odulos referidos por las cl\'ausulas
           \fromImportClause existan y definan las
           rutinas importadas;
  \item[-] que las cl\'ausulas \fromImportClause no sean
           recursivas ni mutuamente recursivas.
           
  \end{itemize}
Si no se cumple alguna de estas condiciones, se eleva una
excepci\'on marcando el \'area del programa donde se
identific\'o el problema.

\subsubsection{An\'alisis de variables vivas}

El m\'odulo \py{lang.gbs_liveness} define la funci\'on \py{check_live_variables},
que toma el AST de un programa y verifica las siguientes condiciones:
  \begin{enumerate}
  \item Si un nodo (del \en{control flow graph}) usa una variable,
        todos los caminos que llegan a dicho nodo deben contener
        una definici\'on de la variable en cuesti\'on.
           
  \item Si un nodo define una variable, debe
        haber alg\'un nodo alcanzable en el cual dicha
        definici\'on se use.
  \end{enumerate}

La condici\'on 2 se relaja haciendo las siguientes consideraciones:
\begin{itemize}
\item[-] Se permite que los par\'ametros de una funci\'on
         o procedimiento no sean utilizados en el cuerpo.
\item[-] Se permite que el \'indice de un \gbs{repeatWith}
         no se use en el cuerpo del ciclo.
\item[-] En una asignaci\'on a una tupla, se exige que
         {\em{al menos una}}, pero {\em{no necesariamente todas}},
         las variables asignadas sean utilizadas.
\end{itemize}

Esto se implementa en dos etapas. En la primera,
se hace el an\'alisis de flujo de datos (\en{backwards}) 
de la manera est\'andar.\footnote{
Algoritmo de an\'alisis de flujo de datos de la Secci\'on 10.6 de [1].}
Cada nodo del AST se anota con
un conjunto de identificadores
\py{live_in} y un conjunto de identificadores \py{live_out}
(conjunto de variables vivas en cada punto del programa).
En la segunda etapa, se recorre el AST y se usa esta
informaci\'on para detectar potenciales errores.

\subsubsection{Inferencia de tipos}

El m\'odulo \py{lang.gbs_infer} provee la funci\'on \py{typecheck}
que toma el AST de un programa \Gbs y hace inferencia de tipos,
elevando una excepci\'on si el programa no es tipable.
Las reglas de tipado son las esperables.

\subsection{Compilaci\'on}

El proceso de compilaci\'on es bastante directo.
Cada rutina (procedimiento o funci\'on) se compila a c\'odigo
para la \GVM. El programa es un diccionario que asocia nombres
de rutinas a su correspondiente c\'odigo compilado.

\subsubsection{Sentencias \gbs{case}}

El \gbs{case} se compila mediante traducci\'on a condicionales,
dejando de lado consideraciones de eficiencia. El siguiente
fragmento:

\begin{verbatim}
    case (Value) of
      Lits1 -> {Body1}
      Lits2 -> {Body2}
      ...
      LitsN -> {BodyN}
      _     -> {BodyElse}
\end{verbatim}
resulta en el mismo c\'odigo que se obtendr\'ia al compilar
un fragmento como el siguiente:
\begin{verbatim}
    v0 := Value
    if      (v0 in Lits1) {Body1}
    else if (v0 in Lits2) {Body2}
    ...
    else if (v0 in LitsN) {BodyN}
    else                  {BodyElse}
\end{verbatim}
Se introduce una variable temporal \gbs{v0},
tomando un nombre \'unico, como t\'ecnica habitual de macroexpansi\'on.
Esto evita que la expresi\'on \gbs{Value} se calcule varias veces.
Notar que, en este caso, evitar evaluaciones m\'ultiples
s\'olo tiene impacto en la eficiencia, y no es indispensable
para la correcci\'on, ya que Gobstones garantiza que la expresi\'on
\gbs{Value} no modifica el estado global, por cuanto da lo
mismo evaluarla una vez o muchas.

\subsubsection{Sentencias \gbs{repeatWith}}

Las sentencias \gbs{repeatWith} tambi\'en se macroexpanden.
El siguiente fragmento:
\begin{verbatim}
    repeatWith i in A..B {BODY}
\end{verbatim}
resulta en el mismo c\'odigo que el siguiente:
\begin{verbatim}
    i := A
    b0 := B
    if (i <= b0) {
      while (True) {
         {BODY}
         if (i == b0) break;
         i := siguiente(i)
      }
    }
\end{verbatim}

\subsection{Int\'erprete de la GVM}


\section{M\'odulo \py{gui}}

\section{Otros}

\subsection{Tableros}

\py{lang.board.formats} -- implementa clases que
permiten leer y escribir tableros en diferentes tipos de
archivos. Los formatos permitidos actualmente son:
\begin{itemize}
\item \textsf{gbt}: formato cl\'asico heredado de la implementaci\'on en Haskell.
\item \textsf{gbb}: formato simple de texto.
\item \textsf{tex}: para exportar tableros e incluirlos en documentos de \LaTeX.
\end{itemize}

\subsection{Juez}
\py{lang.judge}

\appendix
\newpage
\section{Formato de la gram\'atica BNF}

En esta secci\'on se describe informalmente la gram\'atica
reconocida por el generador de \en{parsers}.
La gram\'atica BNF no es realmente \en{parseada} en el sentido
prolijo de la palabra. Consta de una lista de producciones y
definiciones de s\'imbolos terminales. El generador de parsers
levanta un \texttt{.txt} y usa separadores predefinidos
(haciendo \py{split}) para construir, a partir del texto, la
tabla de tokenizaci\'on y el diccionario de producciones.

Se consideran especiales las siguientes secuencias de caracteres:

\begin{verbatim}
        ::=    |    ;;    @    #
\end{verbatim}
Las secuencias de caracteres especiales delimitan partes de
la gram\'atica BNF.

La gram\'atica se escribe mediante un conjunto de cl\'ausulas
separadas por doble punto y coma (\textsc{;;}).
Las cl\'ausulas definen s\'imbolos terminales y no terminales.

Las cl\'ausulas asociadas a s\'imbolos terminales especifican
su sintaxis l\'exica mediante una lista de expresiones regulares
separadas por barras verticales (\verb`|`):

\begin{verbatim}
TERMINAL ::= regexp1 | regexp2 | ... | regexpN ;;
\end{verbatim}
Las expresiones regulares usan la sintaxis del m\'odulo \py{re}
est\'andar de Python.
El nombre de un terminal puede ser cualquier cadena que no
contenga blancos ni secuencias de caracteres especiales.

Las cl\'ausulas asociadas a s\'imbolos no terminales
especifican un conjunto de producciones:

\begin{verbatim}
<non-terminal> ::= prod1 | prod2 | ... | prodN ;;
\end{verbatim}
El nombre de un s\'imbolo no terminal debe empezar con \verb|<|
y terminar con \verb|>| y no debe contener blancos ni
secuencias de caracteres especiales.

Cada producci\'on es una lista de s\'imbolos (terminales o
no terminales) se\-pa\-ra\-dos por espacios.
Opcionalmente, una producci\'on puede estar acompa\~nada
de una acci\'on, que se escribe despu\'es de una \verb|@|.

El no terminal \verb|<start>| es distinguido como
s\'imbolo inicial. Los terminales \verb|WHITESPACE| y \verb|COMMENT|
se ignoran en el an\'alisis l\'exico.

Por ejemplo, la siguiente gram\'atica especifica la sintaxis
de las \en{s-expressions} de Lisp, admitiendo
\verb|a|, \verb|b| y \verb|c| como \'unicos \'atomos.

\begin{verbatim}
<start> ::= <expr> ;;

WHITESPACE ::= [ \t\r\n]+ ;;
LPAREN  ::= \( ;;
RPAREN  ::= \) ;;
atom    ::= a | b | c ;;

<expr>  ::= atom
          | LPAREN <exprs> RPAREN
         ;;
<exprs> ::= <empty>
          | <expr> <exprs>
         ;;
<empty> ::= ;;
\end{verbatim}

\newpage
\section{Sintaxis abstracta de \Gbs}

En esta secci\'on se especifican los ASTs con los que lidia internamente
la implementaci\'on de \PGbs.

\begin{verbatim}
<program>  ::= [ 'program', <imports>, <defs> ]
<imports>  ::= [ <import1>, ..., <importN> ]
<import>   ::= [ 'import', <module>, [ <routine1>, ..., <routineN> ] ]
<module>   ::= uid
<routine>  ::= uid | lid
<defs>     ::= [ <def1>, ..., <defN> ]
<def>      ::= [ 'procedure', uid, [ lid1, ..., lidN ],
                              <commands>, <type_decl> ]
             | [ 'function', lid, [ lid1, ..., lidN ],
                             <commands>, <type_decl> ]
<commands> ::= [ <cmd1>, ..., <cmdN> ]
<cmd> ::= [ 'Skip' ]
        | [ 'THROW_ERROR', str ]
        | [ 'procCall', uid, <args> ]
        | [ 'assignVarName', lid, <expr> ]
        | [ 'assignVarTuple1', [ lid1, ..., lidN ],
                               [ 'funcCall', lid, <args> ] ]
        | [ 'if', <expr>, <block>, None ]      -- sin else
        | [ 'if', <expr>, <block>, <block> ]   -- con else
        | [ 'case', <expr>,
                    <branch1>, ..., <branchN>,
                    [ 'defaultBranch', <block> ] ]
        | [ 'while', <expr>, <block> ]
        | [ 'repeatWith', <expr>, [ 'range', <expr>, <expr> ], <block> ]
        | <block>
        | [ 'return', [ <expr1>, ..., <exprN> ] ]
<block>  ::= [ 'block', <commands> ]
<branch> ::= [ 'branch', [ lit1, ..., litN ], <block> ]
<args>   ::= [ <expr1>, ..., <exprN> ]
<expr>   ::= [ 'or',     op, <expr>, <expr> ]    -- ||
           | [ 'and',    op, <expr>, <expr> ]    -- &&
           | [ 'not',    op, <expr> ]            -- not
           | [ 'relop',  op, <expr>, <expr> ]    -- == /= < <= >= >
           | [ 'addsub', op, <expr>, <expr> ]    -- + -
           | [ 'mul',    op, <expr>, <expr> ]    -- *
           | [ 'divmod', op, <expr>, <expr> ]    -- div mod
           | [ 'pow',    op, <expr>, <expr> ]    -- ^
           | [ 'varName', lid ]                  -- variable
           | [ 'funcCall', lid, <args> ]         -- llamado a función
           | [ 'unaryMinus', op, <expr> ]        -- - (unario)
           | [ 'literal', lit ]
<type_decl> ::= [ 'procType', <types1> ]
              | [ 'funcType', <types1>, <types1> ]
<types1>    ::= [ <type1>, ..., <typeN> ]   -- N >= 1
<type>      ::= uid   -- basic type

str ::= cadena
uid ::= identificador con mayúsculas
lid ::= identificador con minúsculas
lit ::= literal (constante de tipo Bool, Int, Color o Dir)
op  ::= operador
\end{verbatim}

La notaci\'on \py{[x1, ..., xN]} representa un nodo del AST
con \py{N} hijos. Los hijos pueden ser objetos arbitrarios de
Python. La notaci\'on \py{'str'} representa un \en{string}
constante de Python; \py{None} representa dicho valor en Python.
Los identificadores desnudos representan
terminales. Los identificadores envueltos en \verb|<...>| representan
no terminales.

\newpage
\section{\GVM}

\subsection{Estado de la \GVM}

La \GVM es una m\'aquina de dos pilas (pila de valores y \en{call stack}).
El estado actual de la \GVM est\'a dado por:

\begin{itemize}
\item \py{stack}: una pila de valores de \Gbs.
\item \py{ar}: instancia de \py{ActivationRecord}, representa el registro
      de activaci\'on actual. Un registro de activaci\'on contiene:
      \begin{itemize}
      \item programa actual
      \item nombre de la rutina actual
      \item \en{instruction pointer} actual (relativo al \en{bytecode} de la rutina actual)
      \item diccionario de \en{bindings} de variables locales
      \end{itemize}
\item \py{callstack}: pila de llamadas (pila de registros de activaci\'on).
\item \py{global_state}: instancia de \py{GlobalState}, es una
      abstracci\'on del estado global. En nuestro caso, el estado
      global es un tablero.
\end{itemize}

Adicionalmente, en casi todas las estructuras asociadas a fragmentos
de un programa, se mantienen referencias a sus \'arboles sint\'acticos
para permitir el reporte de errores.

\subsection{C\'odigo de la \GVM}
Los opcodes de la \GVM son los siguientes:
\vspace{10pt}

\begin{tabular}{l|l|l}
\textbf{Nombre} & \textbf{Par\'ametros} & \textbf{Efecto en la pila}\\
\hline
pushConst   & const                       &           -- const\\
pushFrom     & var_name                    &           -- var\\
popTo      & var_name                    & value     --\\
call        & rtn_name, nargs             & a1 ... an -- r1 ... rm\\
THROW_ERROR        & str                         &           --\\
label       & label_name                  &           --\\
jump        & label_name                  &           --\\
jumpIfFalse & label_name                  &      cond --\\
jumpIfNotIn & lits, label_name            &     value --\\
return      & nvals                       & a1 ... an -- a1 ... an\\
enter                                   & & -- \\
leave                                   & & -- \\
delVar      & var_name                    & -- \\
\hline
\end{tabular}

\begin{itemize}
\item \py{pushConst const} --
mete en la pila la constante \py{const}.

\item \py{pushFrom var_name} --
mete en la pila el valor de la variable local
llamada \py{var_name}. Debe estar definida en
el diccionario de bindings locales del registro
de activaci\'on \py{ar}.

\item \py{popTo var_name} --
saca de la pila un valor y se lo asigna a la
variable local llamada \py{var_name}, en el
diccionario de bindings locales del registro
de activaci\'on \py{ar}. La variable pod\'ia
no encontrarse anteriormente definida.

\item \py{call rtn_name, nargs} --
saca de la pila \py{nargs} par\'ametros e
invoca a la rutina \py{rtn_name}.
\begin{itemize}
\item si \py{rtn_name} es una rutina \en{builtin},
      calcula el resultado y lo mete en la pila;
\item si \py{rtn_name} es una rutina definida por
      el usuario, guarda el registro de activaci\'on
      actual en la pila de llamadas, actualiza el
      registro de activaci\'on actual para ejecutar
      el c\'odigo correspondiente a la rutina
      dada, y liga las variables locales a los
      par\'ametros obtenidos de la pila.
\end{itemize}
Los par\'ametros en la pila se meten en orden.
Es decir, al momento de hacer un \py{call},
el par\'ametro del tope de la pila es el \'ultimo.

\item \py{THROW_ERROR str} -- autodestrucci\'on con mensaje
de error \py{str}.

\item \py{label label_name} -- pseudoinstrucci\'on,
define la etiqueta \py{label_name} en este punto del
programa.

\item \py{jump label_name} -- salta incondicionalmente
al punto del programa identificado por la etiqueta
\py{label_name}.

\item \py{jumpIfFalse label_name} --
salto condicional. Saca de la pila un valor.
Si dicho valor es la constante \gbs{False} de \Gbs,
salta al punto del programa
identificado por la etiqueta \py{label_name}.

\item \py{jumpIfNotIn lits, label_name} --
salto condicional. Saca de la pila un valor.
Si dicho valor no figura dentro de la lista de
constantes \gbs{lits}, salta al punto del programa
identificado por la etiqueta \py{label_name}.

Esta instrucci\'on se utiliza exclusivamente
para implementar la sentencia \gbs{case}.

\item \py{return nvals} --
retorna de un procedimiento o funci\'on,
sacando un registro de activaci\'on de la
pila de llamadas y estableci\'endolo como
registro de activaci\'on actual.

El n\'umero \py{nvals} indica la cantidad de
valores devueltos, que debe ser 0 para un
procedimiento y un n\'umero positivo para una
funci\'on.

Por convenci\'on, los valores devueltos por una
funci\'on se meten en la pila en el orden en que
figuran en el c\'odigo fuente (de izquierda a derecha)
y se sacan de la pila en el orden inverso.

\item \py{enter} / \py{leave} --
las funciones en Gobstones no deben causar efectos;
sin embargo, el cuerpo de una funci\'on puede contener
llamadas a comandos que s\'i causen efectos sobre el
estado global (i.e. el tablero).

Para respetar la sem\'antica del lenguaje, al
momento de entrar a una funci\'on se guarda
el estado global usando la instrucci\'on
\py{enter}. Al momento de salir de la funci\'on,
se restaura el estado global mediante la instrucci\'on
\py{leave}.

\item \py{delVar var_name} -- elimina el \en{binding}
de la variable local llamada \py{var_name}.

Esta instrucci\'on se utiliza para compilar la sentencia
\gbs{repeatWith}, para eliminar el \en{binding}
del \'indice una vez terminada la repetici\'on.
De lo contrario, el fragmento siguiente
ser\'ia v\'alido:
\begin{verbatim}
    repeatWith i in minColor()..maxColor() {
    }
    Poner(i)
\end{verbatim}

\end{itemize}

%%%%

\newpage
\begin{thebibliography}{BLM05}
\bibitem[1]{}
A. Aho, R. Sethi, J. Ullman.
\newblock {\em Compiladores: Principios, t\'ecnicas y herramientas}.
\end{thebibliography}

  \end{document}
